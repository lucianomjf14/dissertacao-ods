% =============================================================================
% TABELA PARA APÊNDICE DE DISSERTAÇÃO - FORMATO ABNT
% =============================================================================
% 
% INSTRUÇÕES DE USO:
% 1. Adicione no preâmbulo do seu documento LaTeX:
%    \usepackage{longtable}
%    \usepackage{array}
%    \usepackage[brazil]{babel}
%    \usepackage[utf8]{inputenc}
%
% 2. Para usar a tabela compacta (recomendada para paisagem):
%    - Coloque a página em modo paisagem com: \usepackage{rotating}
%    - Use: \begin{landscape} ... \end{landscape}
%
% 3. Para usar a tabela longitudinal (recomendada para retrato):
%    - Pode ser usada normalmente no formato retrato
%
% =============================================================================

% VERSÃO RECOMENDADA: TABELA LONGITUDINAL (FORMATO RETRATO)
% Esta versão é mais adequada para dissertações pois:
% - Quebra automaticamente entre páginas
% - Mantém cabeçalho em todas as páginas
% - Segue normas ABNT para tabelas longas
% =============================================================================

\begin{longtable}{|p{0.8cm}|p{3.5cm}|p{7cm}|p{1cm}|p{2.5cm}|}
\caption{Corpus da análise de conteúdo: artigos selecionados para a pesquisa} 
\label{tab:corpus_analise_conteudo} \\
\hline
\textbf{ID} & \textbf{Autores} & \textbf{Título} & \textbf{Ano} & \textbf{Periódico} \\
\hline
\endfirsthead

\multicolumn{5}{c}%
{\tablename\ \thetable\ -- \textit{Continuação da página anterior}} \\
\hline
\textbf{ID} & \textbf{Autores} & \textbf{Título} & \textbf{Ano} & \textbf{Periódico} \\
\hline
\endhead

\hline 
\multicolumn{5}{r}{\textit{Continua na próxima página}} \\
\endfoot

\hline
\multicolumn{5}{l}{\footnotesize \textbf{Fonte:} Elaborado pelo autor, 2025.} \\
\endlastfoot

1 & SITUM et al. & Smart Mobility in German-Speaking Cities and Sarajevo: Differences, Challenges, Opportunities, and Lessons for Implementation Success & 2024 & Sustainability \\
\hline
2 & LAMDJAD; ALFALAHI & Total Quality Management (TQM) for the development of future smart and integrated cities and sustainable development & 2024 & Journal of Infrastructure, Policy and Development \\
\hline
3 & DAS, 2024 & Exploring the Symbiotic Relationship between Digital Transformation, Infrastructure, Service Delivery, and Governance for Smart Sustainable Cities & 2024 & Smart Cities \\
\hline
4 & ISMAIL; ABDULAZEEZ & Machine Learning Classification Algorithms-Based Smart Cities Applications: A Review & 2024 & The Indonesian Journal of Computer Science \\
\hline
5 & LIFELO et al. & Artificial Intelligence-Enabled Metaverse for Sustainable Smart Cities: Technologies, Applications, Challenges, and Future Directions & 2024 & Electronics \\
\hline
6 & ALNASER; MAXI; ELMOUSALAMI & AI-Powered Digital Twins and Internet of Things for Smart Cities and Sustainable Building Environment & 2024 & Applied Sciences \\
\hline
7 & ZHUANG; CENCI; ZHANG & Review of Big Data Implementation and Expectations in Smart Cities & 2024 & Buildings \\
\hline
8 & GOREN et al. & Recent developments on carbon neutrality through carbon dioxide capture and utilization with clean hydrogen for production of alternative fuels for smart cities & 2024 & International Journal of Hydrogen Energy \\
\hline
9 & CAMACHO et al. & Leveraging Artificial Intelligence to Bolster the Energy Sector in Smart Cities: A Literature Review & 2024 & Energies \\
\hline
10 & MRABET; SLITI & Integrating machine learning for the sustainable development of smart cities & 2024 & Frontiers in Sustainable Cities \\
\hline
11 & ÖZKAYNAK et al. & Neurochallenges in smart cities: state-of-the-art, perspectives, and future directions & 2024 & Frontiers in Neuroscience \\
\hline
12 & VAINIO & Designing technology for smart and sustainable cities of tomorrow & 2024 & The Design Journal \\
\hline
13 & BIBRI et al. & The synergistic interplay of artificial intelligence and digital twins in the built environment: A systematic review and bibliometric analysis & 2024 & Environmental Science and Ecotechnology \\
\hline
14 & ZENG; PANG; TANG & Sensors on Internet of Things Systems for the Sustainable Development of Smart Cities & 2024 & Sensors \\
\hline
15 & ALMEIDA; GUIMARÃES; AMORIM & Exploring the Differences and Similarities between Smart Cities and Sustainable Cities & 2024 & Sustainability \\
\hline
16 & KANTAROS et al. & Leveraging 3D Printing for Resilient Disaster Management in Smart Cities & 2024 & Smart Cities \\
\hline
17 & COSTA et al. & Achieving Sustainable Smart Cities through Geospatial Data-Driven Approaches & 2024 & Sustainability \\
\hline
18 & SHARIFI et al. & Smart cities and sustainable development goals (SDGs): A systematic literature review of co-benefits and trade-offs & 2024 & Cities \\
\hline
19 & IANESE & The role of digital technologies in the development of Smart Cities: an analysis of sustainability aspects & 2023 & Regional Studies and Local Development \\
\hline
20 & ANDRADE et al. & Geotecnologías en el Contexto de las Ciudades Inteligentes: Una Revisión Sistemática de la Literatura & 2023 & Procesos Urbanos \\
\hline
21 & ANDEJANY et al. & Transformation of urban cities to sustainable smart cities-critical success factors and way forward & 2023 & Journal of Theoretical and Applied Information Technology \\
\hline
22 & ISSA ZADEH; GARAY-RONDERO & Enhancing Urban Sustainability: Unravelling Carbon Footprint and Emission Reduction Strategies for Smart Cities & 2023 & Smart Cities \\
\hline
23 & BIBRI et al. & Environmentally sustainable smart cities and their converging AI, IoT, and big data technologies and solutions: an integrated approach to an extensive literature review & 2023 & Energy Informatics \\
\hline
24 & BARRETO; QUINTELLA & Transporte Hidroviário: uma análise de Revisão Sistemática Para Auxílio ao Desenvolvimento de Cidades Inteligentes & 2023 & Cadernos de Prospecção \\
\hline
25 & KAGINALKAR et al. & SmartAirQ: A Big Data Governance Framework for Urban Air Quality Management in Smart Cities & 2022 & Frontiers in Environmental Science \\
\hline
26 & ALSHUWAIKHAT; AINA; BINSAEDAN & Analysis of the implementation of urban computing in smart cities & 2022 & Heliyon \\
\hline
27 & BURLACU; BOBOC; BUTILĂ & Smart Cities and Transportation: Reviewing the Scientific Character of the Theories & 2022 & Sustainability \\
\hline
28 & ALLAM et al. & The Metaverse as a Virtual Form of Smart Cities: Opportunities and Challenges for Environmental, Economic, and Social Sustainability & 2022 & Smart Cities \\
\hline
29 & BELLINI; NESI; PANTALEO & IoT-Enabled Smart Cities: A Review of Concepts, Frameworks and Key Technologies & 2022 & Applied Sciences \\
\hline
30 & JERE et al. & An Evaluation of Developing Smart Cities in Developing Countries & 2022 & Zambia ICT Journal \\
\hline
31 & HERATH; MITTAL & Adoption of artificial intelligence in smart cities: A comprehensive review & 2022 & International Journal of Information Management \\
\hline
32 & ALMIHAT et al. & Energy and Sustainable Development in Smart Cities: An Overview & 2022 & Smart Cities \\
\hline
33 & SENGUPTA; SENGUPTA & SDG-11 and smart cities: Contradictions and overlaps between sustainability and smartness & 2022 & Frontiers in Sociology \\
\hline
34 & KASINATHAN et al. & Realization of Sustainable Development Goals with Disruptive Technologies by Integrating Industry 5.0, Society 5.0, Smart Cities and Villages & 2022 & Sustainability \\
\hline
35 & CATALANO et al. & Smart Sustainable Cities of the New Millennium: Towards Design for Nature & 2021 & Circular Economy and Sustainability \\
\hline
36 & ROCHA et al. & Smart Cities' Applications to Facilitate the Mobility of Older Adults: A Systematic Review & 2021 & Applied Sciences \\
\hline
37 & MASTRODI; BROLLO; RIBEIRO & A GOVERNANÇA E A GESTÃO COMO ESTRATÉGIAS DE INCLUSÃO NAS CIDADES INTELIGENTES & 2021 & Juris Poiesis - Qualis B1 \\
\hline
38 & BELLI et al. & IoT-Enabled Smart Sustainable Cities: Challenges and Approaches for Developing Countries-A Case Study of Florianópolis, Brazil & 2020 & Smart Cities \\
\hline
39 & ANDRADE et al. & A Comprehensive Study of the IoT Cybersecurity in Smart Cities & 2020 & IEEE Access \\
\hline
40 & SUKHWANI et al. & Role of Smart Cities in Optimizing Water-Energy-Food Nexus: Opportunities in Nagpur & 2020 & Smart Cities \\
\hline
41 & GIMPEL et al. & Information Systems for Sustainable Use of Water in Smart Cities & 2020 & Pre-ICIS Workshop Proceedings 2020 \\
\hline
\end{longtable}

% =============================================================================
% EXEMPLO DE COMO REFERENCIAR A TABELA NO TEXTO:
% =============================================================================
%
% O corpus da pesquisa foi constituído por 41 artigos científicos selecionados 
% conforme os critérios de inclusão e exclusão estabelecidos na metodologia, 
% conforme apresentado na Tabela \ref{tab:corpus_analise_conteudo} (Apêndice A).
% Os artigos abrangem o período de 2020 a 2024, demonstrando a relevância 
% contemporânea do tema de cidades inteligentes sustentáveis.
%
% =============================================================================
